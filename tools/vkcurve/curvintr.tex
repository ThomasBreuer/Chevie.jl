%%%%%%%%%%%%%%%%%%%%%%%%%%%%%%%%%%%%%%%%%%%%%%%%%%%%%%%%%%%%%%%%%%%%%%%%%
%%
%A  curvintr.tex      VKCURVE documentation    David Bessis,  Jean Michel
%%
%Y  Copyright (C) 2001-2002  University  Paris VII.
%%
%%  This  file  introduces the VKCURVE package.
%%
%%%%%%%%%%%%%%%%%%%%%%%%%%%%%%%%%%%%%%%%%%%%%%%%%%%%%%%%%%%%%%%%%%%%%%%%%
\def\VKCURVE{{\sf VKCURVE}}
\def\CHEVIE{{\sf CHEVIE}}
\Chapter{The VKCURVE package}

The  main function of the \VKCURVE\  package computes the fundamental group
of  the  complement  of  a  complex  algebraic  curve  in  $\C^2$, using an
implementation  of the Van Kampen method  (see for example \cite{C73} for a
clear and modernized account of this method).

|    gap> FundamentalGroup(x^2-y^3);
    #I  there are 2 generators and 1 relator of total length 6
    1: bab=aba
    
    gap> FundamentalGroup((x+y)*(x-y)*(x+2*y));
    #I  there are 3 generators and 2 relators of total length 12
    1: cab=abc
    2: bca=abc|

The  input is  a  polynomial in  the  two variables  'x'  and 'y',  with
rational  coefficients.  Though  approximate calculations  are  used  at
various places, they are controlled and the final result is exact.

The output is a  record which  contains lots  of information  about the
computation, including a presentation of the computed fundamental group,
which is what is displayed when printing the record.

Our  motivation   for  writing  this   package  was  to   find  explicit
presentations for  generalized braid groups attached  to certain complex
reflection groups. Though presentations were known for almost all cases,
six exceptional  cases were  missing (in the  notations of  Shephard and
Todd, these  cases are $G_{24}$, $G_{27}$,  $G_{29}$, $G_{31}$, $G_{33}$
and $G_{34}$).  Since the a  priori existence of nice  presentations for
braid groups was proved in \cite{B01}, it was upsetting not to know them
explicitly. In the absence of any good grip on the geometry of these six
examples, brute force  was a way to  get an answer. Using  \VKCURVE , we
have  obtained  presentations for  all of them.

This package was developed thanks to computer resources of the Institut
de Math\accent 19  ematiques de Jussieu in Paris. We  thank the computer
support team,  especially Jo\accent 127  el Marchand, for  the stability
and the efficiency of the working environment.

We have tried to design this package with the novice \GAP\ user in mind.
The only steps required to use it are
\begin{itemize}
\item Run \GAP 3 (the package is not compatible with \GAP 4).
\item  Make  sure  the  packages   \CHEVIE\  and  \VKCURVE\  are  loaded
(beware   that  we   require   the  development   version  of   \CHEVIE,
'http\://www.math.jussieu.fr/\~{}jmichel/chevie.html' and not the one in
the \GAP.3.3.4 distribution)

\item Use the function 'FundamentalGroup',  as demonstrated in the above
examples.

\end{itemize}
If  you are  not interested  in  the details  of the  algorithm, and  if
'FundamentalGroup' gives you satisfactory  answers in a reasonable time,
then you do not need to read this manual any further.

\bigskip

We  use our  own  package  for multivariate  polynomials  which is  more
effective, for  our purposes, than  the default  in \GAP 3  (see 'Mvp').
When \VKCURVE\ is  loaded, the variables 'x' and 'y'  are pre-defined as
'Mvp's; one can  also use \GAP\ polynomials (which will  be converted to
'Mvp's).

The  implementation uses 'Decimal' numbers, 'Complex' numbers and braids as
implemented  in  the  (development  version  of  the)  package  \CHEVIE, so
\VKCURVE\ is dependent on this package.

To implement  the algorithms, we  needed to write  auxiliary facilities,
for instance find  zeros of complex polynomials, or  work with piecewise
linear  braids,  which  may  be  useful  on  their  own.  These  various
facilities are documented in this manual.

Before discussing  our actual  implementation, let  us give  an informal
summary of the mathematical background. Our strategy is adapted from the
one originally  described in the  1930\'s by Van  Kampen. Let $C$  be an
affine  algebraic curve,  given as  the  set of  zeros in  $\C^2$ of  a
non-zero  reduced  polynomial $P(x,y)$.  The  problem  is to  compute  a
presentation of the fundamental group of  $\C^2 - C$. Consider $P$ as a
polynomial in $x$, with coefficients in the ring of polynomials in $y$
$$P= \alpha_0(y)x^n +  \alpha_1(y) x^{n-1}  + \dots +  \alpha_{n-1}(y) x
+  \alpha_n(y),$$  where the  $\alpha_i$  are  polynomials in  $y$.  Let
$\Delta(y)$ be the discriminant of $P$ or, in other words, the resultant
of  $P$  and $\frac{\partial  P}{\partial  x}$.  Since $P$  is  reduced,
$\Delta$ is non-zero. For a generic  value of $y$, the polynomial in <x>
given by $P(x,y)$ has $n$ distinct roots.
When $y=y_j$, with $j$ in $1,\dots,d$,
we are in exactly one of
the following situations\:\ either $P(x,y_j)=0$
(we then say that $y_j$ is bad),
or $P(x,y_j)$ has a number of roots in $x$ strictly smaller than
$n$.
Fix $y_0$  in $\C  - \{y_1,\dots,y_d\}$.  Consider the  projection $p\:
\C^2  \rightarrow \C,  (x,y) \mapsto  y$.  It restricts  to a  locally
trivial  fibration with  base  space $B=  \C  - \{y_1,\dots,y_d\}$  and
fibers homeomorphic  to the  complex plane with  $n$ points  removed. We
denote by  $E$ the  total space  $p^{-1}(B)$ and by  $F$ the  fiber over
$y_0$. The fundamental  group of $F$ is isomorphic to  the free group on
$n$ generators.  Let $\gamma_1,\dots,\gamma_d$  be loops in  the pointed
space  $(B,y_0)$ representing  a generating  system for  $\pi_1(B,y_0)$.
By  trivializing  the pullback  of  $p$  along  $\gamma_i$, one  gets  a
(well-defined up to isotopy) homeomorphism  of $F$, and a (well-defined)
automorphism  $\phi_i$  of  the  fundamental group  of  $F$,  identified
with  the  free  group  $F_n$  by the  choice  of  a  generating  system
$f_1,\dots,f_n$. An effective way of  computing $\phi_i$ is by following
the solutions in $x$ of $P(x,y)=0$,  when $y$ moves along $\phi_i$. This
defines a loop in  the space of configuration of $n$  points in a plane,
hence an element  $b_i$ of the braid group $B_n$  (via an identification
of $B_n$  with the fundamental  group of this configuration  space). Let
$\phi$ be the Hurwitz action of $B_n$  on $F_n$. All choices can be made
in such a way that $\phi_i=\phi(b_i)$. The theorem of Van Kampen asserts
that, if there are no bad  roots of the discriminant, a presentation for
the fundamental group of $\C^2 -  C$ is $$\< f_1,\dots,f_n \mid \forall
i,j,  \phi_i(f_j)=f_j >  $$ A  variant  of the  above presentation  (see
'VKQuotient') can be used to deal with bad roots of the discriminant.

This algorithm is implemented in the following way.

\begin{itemize}
\item As input,  we have a polynomial $P$. The  polynomial is reduced if
it was not.

\item The discriminant $\Delta$ of $P$  with respect to $x$ is computed.
It is a polynomial in $y$.

\item  The  roots  of  $\Delta$  are  approximated,  via  the  following
procedure. First, we reduce  $\Delta$ and get $\Delta_{red}$ (generating
the   radical  of   the  ideal   generated  by   $\Delta$).  The   roots
$\{y_1,\dots,y_d\}$ of  $\Delta_{red}$ are separated  by 'SeparateRoots'
(which implements Newton\'s method).

\item Loops  around these roots are  computed by 'LoopsAroundPunctures'.
This function first computes some sort of honeycomb, consisting of a set
$S$  of  affine  segments,  isolating  the $y_i$.  Since  it  makes  the
computation of  the monodromy  more effective, each  inner segment  is a
fragment of the mediatrix of two roots of $\Delta$. Then a vertex of one
the segments is  chosen as a basepoint, and the  function returns a list
of lists of  oriented segments in $S$\:\ each list  of segment encodes a
piecewise linear loop $\gamma_i$ circling one of $y_i$.

\item For each  segment in $S$, we compute the  monodromy braid obtained
by  following  the  solutions  in  $x$  of  $P(x,y)=0$  when  $y$  moves
along  the segment.  By default,  this  monodromy braid  is computed  by
'FollowMonodromy'. The  strategy is to compute  a piecewise-linear braid
approximating the  actual monodromy geometric braid.  The approximations
are controlled. The piecewise-linear  braid is constructed step-by-step,
by computations of linear pieces. As soon as new piece is constructed, it
is converted into an element  of $B_n$ and multiplied; therefore, though
the  braid  may  consist  of  a huge  number  of  pieces,  the  function
'FollowMonodromy' works with constant memory. The packages also contains
a  variant  function  'ApproxFollowMonodromy', which  runs  faster,  but
without guarantee on the result (see below).

\item The monodromy  braids $b_i$ corresponding to  the loops $\gamma_i$
are  obtained  by  multiplying  the corresponding  monodromy  braids  of
segments. The action of these elements  of $B_n$ on the free group $F_n$
is  computed  by 'BnActsOnFn'  and  the  resulting presentation  of  the
fundamental group is computed by 'VKQuotient'. It happens for some large
problems that  the whole fundamental  group process fails  here, because
the braids $b_i$ obtained are too long and the computation of the action
on $F_n$ requires thus too much memory.  We have been able to solve such
problems  when they  occur by  calling on  the $b_i$  at this  stage our
function 'ShrinkBraidGeneratingSet'  which finds smaller  generators for
the subgroup of $B_n$ generated by the $b_i$ (see the description in the
third chapter). This  function is called automatically at  this stage if
'VKCURVE.shrinkBraid' is set to 'true' (the default for this variable is
'false').

\item Finally,  the presentation is simplified  by 'ShrinkPresentation'.
This  function is  a heuristic  adaptation and  refinement of  the basic
\GAP\ functions for simplifying presentations. It is non-deterministic.

\end{itemize}

From the algorithmic  point of view, memory should not  be an issue, but
the procedure may  take a lot of  CPU time (the critical  part being the
computation of the monodromy braids by 'FollowMonodromy'). For instance,
an empirical  study with the  curves $x^2-y^n$ suggests that  the needed
time grows  exponentially with  $n$. Two solutions  are offered  to deal
with curves for which the computation time becomes unreasonable.

A   global  variable  'VKCURVE.monodromyApprox'  controls  which  monodromy
function  is used.  The default  value of  this variable  is 'false', which
means  that 'FollowMonodromy' will be  used. If the variable  is set by the
user  to  'true'  then  the  function  'ApproxFollowMonodromy' will be used
instead.   This  function  runs  faster  than  'FollowMonodromy',  but  the
approximations  are no longer  controlled. Therefore presentations obtained
while  'VKCURVE.monodromyApprox'  is  set  to  'true'  are  not  certified.
However,  though  it  is  likely  that  there  exists  examples  for  which
'ApproxFollowMonodromy'  actually returns incorrect  answers, we still have
not seen one.

The second way of dealing with  difficult examples is to parallelize the
computation. Since  the computations  of the  monodromy braids  for each
segment  are  independent,  they  can  be  performed  simultaneously  on
different computers. The functions 'PrepareFundamentalGroup', 'Segments'
and   'FinishFundamentalGroup'  provide   basic  support   for  parallel
computing.

%%%%%%%%%%%%%%%%%%%%%%%%%%%%%%%%%%%%%%%%%%%%%%%%%%%%%%%%%%%%%%%%%%%%%%%%%%%%
\Section{FundamentalGroup}
\index{FundamentalGroup}

'FundamentalGroup(<curve> [, <printlevel>])'

<curve> should be an 'Mvp' in <x>  and <y>, or a \GAP\ polynomial in two
variables (which means a polynomial in a variable which is assumed to be
'y' over the polynomial ring $\Q[x]$) representing an equation $f(x,y)$
for a curve  in $\C^2$. The coefficients should  be rationals, gaussian
rationals or 'Complex' rationals. The result  is a record with a certain
number of fields which record steps in the computation described in this
introduction\:

|    gap> r:=FundamentalGroup(x^2-y^3);
    #I  there are 2 generators and 1 relator of total length 6
    1: bab=aba
    
    gap> RecFields(r);
    [ "curve", "discy", "roots", "dispersal", "points", "segments", "loops",
      "zeros", "B", "monodromy", "basepoint", "dispersal", "braids", 
      "presentation","operations" ]
    gap> r.curve;
    x^2-y^3
    gap> r.discy;
    X(Rationals)
    gap> r.roots;
    [ 0 ]
    gap> r.points;
    [ -I, -1, 1, I ]
    gap> r.segments;
    [ [ 1, 2 ], [ 1, 3 ], [ 2, 4 ], [ 3, 4 ] ]
    gap> r.loops;
    [ [ 4, -3, -1, 2 ] ]
    gap> r.zeros;
    [ [ 707106781187/1000000000000+707106781187/1000000000000I,
       -707106781187/1000000000000-707106781187/1000000000000I ],
      [ I, -I ], [ 1, -1 ],
      [ -707106781187/1000000000000+707106781187/1000000000000I,
      707106781187/1000000000000-707106781187/1000000000000I ] ]
    gap> r.monodromy;
    [ (w0)^-1, w0, , w0 ]
    gap> r.braids;
    [ w0.w0.w0 ]
    gap> DisplayPresentation(r.presentation);
    1: bab=aba|

Here 'r.curve' records the  entered equation, 'r.discy' its discriminant
with  respect  to  <x>,  'r.roots'   the  roots  of  this  discriminant,
'r.points',  'r.segments' and  'r.loops'  describes  loops around  these
zeros  as  explained  in   the  documentation  of  'LoopsAroundPunctures';
'r.zeros'  records the  zeros of  $f(x,y_i)$  when $y_i$  runs over  the
various 'r.points';  'r.monodromy' records  the monodromy along  each of
'r.segments', and 'r.braids' is the resulting monodromy along the loops.
Finally 'r.presentation'  records the  resulting presentation  (which is
what is printed by default when 'r' is printed).

The second optional argument triggers  the display of information on the
progress of the  computation. It is recommended to  set the <printlevel>
at 1 or 2  when the computation seems to take a  long time without doing
anything. <printlevel> set  at 0 is the default and  prints nothing; set
at 1 it shows which segment is  currently active, and set at 2 it traces
the computation inside each segment.

|    gap> FundamentalGroup(x^2-y^3,1);
    # There are 4 segments in 1 loops
    # The following braid was computed by FollowMonodromy in 8 steps.
    monodromy[1]:=B(-1);
    # segment 1/4 Time=0sec
    # The following braid was computed by FollowMonodromy in 8 steps.
    monodromy[2]:=B(1);
    # segment 2/4 Time=0sec
    # The following braid was computed by FollowMonodromy in 8 steps.
    monodromy[3]:=B();
    # segment 3/4 Time=0sec
    # The following braid was computed by FollowMonodromy in 8 steps.
    monodromy[4]:=B(1);
    # segment 4/4 Time=0sec
    # Computing monodromy braids
    # loop[1]=w0.w0.w0
    #I  there are 2 generators and 1 relator of total length 6
    1: bab=aba|

%%%%%%%%%%%%%%%%%%%%%%%%%%%%%%%%%%%%%%%%%%%%%%%%%%%%%%%%%%%%%%%%%%%%%%%%%
\Section{PrepareFundamentalGroup}
\index{PrepareFundamentalGroup}

'PrepareFundamentalGroup(<curve>, <name>)'

'VKCURVE.Segments(<name>[,<range>])'

'FinishFundamentalGroup(<r>)'
\index{FinishFundamentalGroup}

These  functions provide  a means  of distributing  a fundamental  group
computation over  several machines.  The basic strategy  is to  write to
a  file  the  startup-information  necessary to  compute  the  monodromy
along  a  segment,  in  the   form  of  a  partially-filled  version  of
the  record returned  by  'FundamentalGroup'. Then  the monodromy  along
each  segment can  be  done in  a separate  process,  writing again  the
result  to files.  These  results  are then  gathered  and processed  by
'FinishFundamentalGroup'. The whole process is illustrated in an example
below.  The  extra argument  <name>  to  'PrepareFundamentalGroup' is  a
prefix used to name intermediate files. One does first \:

|    gap> PrepareFundamentalGroup(x^2-y^3,"a2");
        ----------------------------------
    Data saved in a2.tmp
    You can now compute segments 1 to 4
    in different GAP sessions by doing in each of them:
        a2:=rec(name:="a2");
        VKCURVE.Segments(a2,[1..4]);
    (or some other range depending on the session)
    Then when all files a2.xx have been computed finish by
        a2:=rec(name:="a2");
        FinishFundamentalGroup(a2);|

Then  one can  compute in  separate  sessions the  monodromy along  each
segment.  The second  argument  of 'Segments'  tells  which segments  to
compute in the current session (the  default is all). An example of such
sessions may be\:

|    gap> a2:=rec(name:="a2");
    rec(
      name := "a2" )
    gap> VKCURVE.Segments(a2,[2]);
    # The following braid was computed by FollowMonodromy in 8 steps.
    a2.monodromy[2]:=a2.B(1);
    # segment 2/4 Time=0.1sec
    gap> a2:=rec(name:="a2");
    rec(
      name := "a2" )
    gap> VKCURVE.Segments(a2,[1,3,4]);
    # The following braid was computed by FollowMonodromy in 8 steps.
    a2.monodromy[2]:=a2.B(1);
    # segment 2/4 Time=0.1sec|

When all segments have been computed the final session looks like:

|    gap> a2:=rec(name:="a2");
    rec(
      name := "a2" )
    gap> FinishFundamentalGroup(a2);
    1: bab=aba|

%%%%%%%%%%%%%%%%%%%%%%%%%%%%%%%%%%%%%%%%%%%%%%%%%%%%%%%%%%%%%%%%%%%%%%%%%
